\documentclass[11pt]{article}
\usepackage[a4paper, margin=1in]{geometry}
\usepackage{pgfgantt}


\title{CM50175 – Project Proposal}
\author{Oscar Dos Santos Nunes \\ Student ID: JOFDSN20}
\date{28 March 2025}

\begin{document}
\maketitle

\section*{Project Title}
The current title for the project is \textbf{"Automated Essay Scoring Using Deep Learning with XXX"}.

This title reflects to general aim for the work to design, implement and evaluate and automated essay scoring (AES) 
algorithm focusing on the utilization of deep learning techniques.

However, the title will be updated as the project progresses and the research focus becomes more defined. 
The current placeholder “XXX” represents a yet to be decided add-on. 
This may involve the implementation of a specific deep-learning architecture—such as LSTM, 
BERT or a hybrid transformers model or the integration of additional functionality such as expandability, 
feedback generation, or the processing of handwritten text.

\vspace{0.5em}

Some possible titles include:
\begin{itemize}
    \item \textit{Automated Essay Scoring Using Deep Learning}
    \item \textit{Neural Approaches to Automated Essay Scoring: A Deep Learning Perspective}
    \item \textit{Transformer-Based Models for Automated Essay Scoring}
    \item \textit{Evaluating Student Writing with Deep Learning: A Comparative Approach}
    \item \textit{Automated Essay Scoring with Explainable Deep Learning Architectures}
    \item \textit{Beyond Accuracy: Fair and Interpretable Deep Learning for Essay Scoring}
    \item \textit{Multi-modal Essay Assessment: Integrating Handwritten and Typed Text in Neural AES Systems}
    \item \textit{Benchmarking Traditional and Deep Learning Models for Essay Scoring}
\end{itemize}


\section*{Problem Statement}
Essay based work is an essential part of the educational process and therefore is vital for performance evaluation. However
the grading of essays is a highly subjective process and can be time consuming. Studies have shown that the grading of essays
by different teachers can vary significantly, with some teachers being more lenient than others. This can lead to inconsistencies
in grading and can affect the overall performance of students. These issues are especially prevalent in larger-scale assessments 
(e.g. GCSEs, ALevels and Uni assessments) where many markers are involved. The use of automated essay scoring (AES) systems can help 
to address these issues by providing a more consistent and objective grading process.

AES systems aim to address these limitations through computational techniques to evaluate the quality of written text. Early AES systems 
such as Project Essay Grade, relied very heavily on manually engineered features such as grammar usage, word length, and sentence structure, combined with
traditional machine learning models. While these systems showed success (e-rater was used in english as a foreign language scoring), they often lacked
robustness and generalization capabilities.

Recent advancements in deep learning and natural language processing have introduced more sophisticated models and approaches to AES.
Neural architectures such as LSTM networks and transformers based models such as BERT have achieved high performance across a variety of NLP tasks, 
including text classification and sentiment analysis. For example, BERT achieved a GLUE benchmark score of 80.5 in its original implementation, showing its
ability to to model contextual information and the relationships between words in a sentence. The models are well suited to AES tasks due to their ability to 
contextualize meaning beyond the surface-level features.

However, there are still several challenges to overcome. Deep networks often act as "black boxes", which offer very little interpretability into how they arrive
at an output. This creases a concern for use in education where expandability/feedback and essential. Furthermore, these models require vast quantities of 
labeled-data of which is not always available. Additionally there are a number of documented risks regarding the algorithmic bias in AES particularly when
dealing with non-native speakers or student with a less represented background than in the training data.

Moreover, much of the existing AES models focus solely on predicting a final holistic score, with less of a focus on providing additives that would be given by a human marker,
such as feedback on specific aspects, annotated marking , or the ability to discuss the current mark and how to improve. Each of these would substantially improve the
possible use of AES in education.

This project aims to explore and evaluate the application of deep learning techniques to automate essay scoring, which a focus on performance, fairness, and extensibility. Through the implementation of
a deep learning model, the research aims to contribute to the development of a more accurate, explainable and general use AES system.


\section*{Objectives and Research Questions}
This project aims to investigate the use of deep learning for automated essay scoring, focusing on the development of a model that
is accurate, fair and explainable. The ultimate goal is the develop a system that can generate consistent
and reliable scoring for comparable to that of human marking, while also considering the systems interpretability and the 
potential for future enhancements such as feedback generation or support for handwritten text

\subsection{Objectives}
\begin{enumerate}
    \item \textbf{Review Existing AES Systems And literature:} \textit{Identify the strengths, limitations, and gaps in current approaches,
    particularly in those using deep learning techniques.}
    \item \textbf{Select And Prepare Datasets:} \textit{Selecting, preparing and pre-processing datasets for training and evaluation of the model.}
    \item \textbf{Develop And Compare Model Performance:} \textit{Implementing and evaluating a deep learning model for AES,
    comparing its performance to traditional machine learning approaches.}
    \item \textbf{Evaluate The Performance, Fairness And Interpretability:} \textit{Investigate any issues with Biases, inconsistencies in the methods, as well as
    the models transparency and "trustworthiness".}
    \item \textbf{Explore Potential Enhancements:} \textit{Investigate the potential for future enhancements to the model, such as feedback generation, or multi-modal input.}

\end{enumerate}

\subsection{Research Questions}
\begin{enumerate}
    \item To what extent can transformer-based models outperform traditional machine learning approaches within the context of an automated essay scoring system?
    \item What types of biases are present in AES systems, and how can deep learning models be adapted to mitigate these issues?
    \item what are the potential trade offs between scoring accuracy, computational efficiency, and interpretability within the current AES techniques?
    \item Can AES models by extended to provide meaningful and human-like feedback to support their usage within the educational setting?
\end{enumerate}

\section*{Background and Related Work}
Automated Essay Scoring (AES) systems have been in development for over five decades, evolving from rule-based and statistical 
approaches into increasingly complex machine learning and deep learning technologies. Early developments such as Project Essay 
Grade (PEG) and e-rater relied on handcrafted features, including high-level syntactic and lexical characteristics. While these 
methods demonstrated some success and saw real-world application they often struggled to capture the semantic and contextual 
depth found in students work.

With the growth of more data focused approaches in natural language processing (NLP), AES has increasingly adopted machine learning 
techniques. Traditional models such as support vector machines (SVM), random forests, and ridge regression were trained on 
handcrafted features to predict human-marked essay scores. However, these models were still limited by their reliance on manually 
created inputs, which creates constraints with the scalability and generalization across writing prompts and student population 
groups.

The growth of deep learning, particularly with transformer-based models, has significantly advanced AES research. Pre-trained 
language models such as \textbf{BERT} and \textbf{RoBERTa} have demonstrated strong performance across a wide range of NLP tasks, thanks to their 
ability to model contextual dependencies within text. When compared to more typical machine learning techniques, studies using BERT 
for AES have shown notable gains in predicting accuracy, especially when using benchmark datasets such as the \textbf{ASAP (Automated 
Student Assessment Prize) corpus}.

Despite these advancements, a number of obstacles still exists, deep learning models frequently operate as "black boxes", providing little room 
for interpretation. This is a major drawback for AES, because user trust and educational integration rely on score reasoning and openness.
Second, there are now worries about algorithmic bias, especially when it comes to under-represented student groups or non-native english speakers.
These issues raise questions about fairness, generalisability, and the ethical use of AES in high-stakes contexts. These issues raise questions about 
fairness, generalisability, and the ethical use of AES in high-stakes contexts.

Furthermore, much of current AES work focuses on holistic scoring without exploring the formative aspects of feedback, rubric-based scoring, or the 
integration of multi-modal inputs. Expandability methods such as SHAP or attention-based visualizations have had lots of research recently to focus
on improving the models transparency, while hybrid systems that combine neural and symbolic components have showed promise in improving interpretability.
and control.

This project builds on these advancements by developing an AES model using deep learning, evaluating its performance and fairness, and exploring potential 
avenues for expansion. By benchmarking against traditional baselines and considering explainability and ethical implications, this work aims to contribute  
to the growing body of research seeking to make AES systems more robust, transparent, and educationally useful.

\section*{Project Plan}
\subsection*{Task Breakdown}
Bellow outlines the key tasks associated with the project, along with their expected duration for completion. Important to note that week 1 is considered 
as the week commencing March 31\textsuperscript{st}.

\begin{center}
    \renewcommand{\arraystretch}{1.5}
    \begin{tabular}{|c|p{5cm}|c|p{6cm}|} 
        \hline
        \textbf{Task ID} & \textbf{Task Description} & \textbf{Duration (Weeks)} & \textbf{Deliverables} \\ 
        \hline
        T1 & Literature review and intro sections & Weeks 1 - 4 & Report sections completed and ready for submission \\ 
        \hline
        T2 & Dataset pre-processing and model development plan & Weeks 3 - 6 & Detailed plan for model development and datasets ready for next steps \\ 
        \hline
        T3 & Baseline model performance evaluation & Weeks 6 - 9 & Use related models and classic ML techniques to establish a baseline for performance \\ 
        \hline
        T4 & Deep Learning model development & Weeks 8 - 14 & Using related work and other models, develop out own AES system \\ 
        \hline
        T5 & Evaluation; performance, fairness, and explainability & Weeks 14 - 17  & Document models performance against key metrics and other models \\ 
        \hline
        T6 & Reporting and documentation & Weeks 18 - 21 & Complete the report and discuss the optional extensions \\ 
        \hline
        T7 & System extension development & Weeks 20 - 22 & Develop additional systems and discuss changes to performance \\ 
        \hline
    \end{tabular}
    \end{center}

\subsection*{Timeline}

\resizebox{\textwidth}{!}{
\begin{ganttchart}[
    hgrid,
    vgrid,
    title/.style={draw=none, fill=none},
    bar height=0.6,
    bar label font=\small\bfseries
    ]{1}{22}

    \gantttitle{Project Timeline (Weeks 1–22)}{22} \\
    \gantttitlelist{1,...,22}{1} \\

    \ganttbar{T1: Literature Review}{1}{4} \\
    \ganttbar{T2: Dataset + Plan}{3}{6} \\
    \ganttbar{T3: Baseline Models}{6}{9} \\
    \ganttbar{T4: DL Model Development}{8}{14} \\
    \ganttbar{T5: Evaluation + Explainability}{14}{17} \\
    \ganttbar{T6: Report + Docs}{18}{21} \\
    \ganttbar{T7: Extension Work}{20}{22}
\end{ganttchart}
}

\subsection*{Dependencies and Risks}

This projects tasks are sequential in nature, with each task building upon the findings of the previous one. due to that I have decided to omit a discussion for the dependencies. However 
below is an outline of the potential risks and the steps that will be taken to mitigate their effects.


\begin{center}
    \renewcommand{\arraystretch}{1.4}
    \begin{tabular}{|p{4.5cm}|p{6cm}|p{4.5cm}|}
        \hline
        \textbf{Risk} & \textbf{Impact} & \textbf{Mitigation Strategy} \\
        \hline
        Limited access to large, well-labelled datasets & May reduce model performance and generalisability & Use multiple publicly available datasets; explore data augmentation techniques; optionally 
        explore synthetic data generation using GenAI \\
        \hline
        Inconsistent scoring methods and scales across datasets & May skew model results and reduce comparability across sources & Implement stratified sampling, resampling, or weighted loss functions 
        to normalise scoring differences \\
        \hline
        Lack of experience with deep learning and transformer architectures & Poor model performance; longer development time & Allocate early project time to tutorials and documentation; replicate 
        open-source AES models to build understanding \\
        \hline
        Deep learning model underperforms compared to baselines & Model may not surpass traditional ML methods & Clearly document findings and contribute insight; maximise use of university resources 
        to improve model tuning and evaluation \\
        \hline
        Computational constraints & Prolonged training times or restricted model size & Leverage University GPU clusters and free-tier HPC resources such as Google Colab Pro or Kaggle Kernels \\
        \hline
        Ethical concerns related to student data & Potential breaches of data usage policy or ethical standards & Confirm dataset licences and ethics approvals early; ensure anonymisation where applicable \\
        \hline
        Falling behind schedule & Risk of incomplete implementation or rushed final report & Follow Gantt milestones; reserve final weeks for catch-up if needed; adjust non-core goals accordingly \\
        \hline
        Scope creep & Attempting too many extensions beyond core AES system & Limit focus to clearly defined objectives; only pursue optional enhancements once core implementation is complete \\
        \hline
    \end{tabular}
    \end{center}


\section*{Resources and Limitations}

This project will draw upon a combination of computational, data, and academic resources available through the University of Bath and open-source platforms.

\subsection*{Resources}

\begin{itemize}
    \item \textbf{Datasets:} Publicly available Automated Essay Scoring datasets, particularly the ASAP (Automated Student Assessment Prize) dataset hosted on Kaggle. Other datasets may be used to 
    supplement or test model generalisability, depending on availability.
    
    \item \textbf{Computational Resources:} Access to University of Bath’s High-Performance Computing (HPC) infrastructure, including GPU support where possible. Cloud services such as Google Colab 
    or Kaggle Kernels may also be used for experimentation or prototyping.
    
    \item \textbf{Libraries and Tools:} Python-based machine learning and NLP libraries, including Hugging Face Transformers, Scikit-learn, PyTorch or TensorFlow (depending on final implementation), 
    and supporting libraries for data processing and visualisations.
    
    \item \textbf{Version Control and Documentation:} GitHub will be used for version control, issue tracking, and collaboration (should it be used anywhere). LaTeX will be used for academic report writing and 
    formatting.
    
    \item \textbf{Academic Resources:} Access to online academic journals, papers, and resources via the University of Bath library (e.g., IEEE Xplore, ACL Anthology, JSTOR).
\end{itemize}

\subsection*{Limitations}

\begin{itemize}
    \item \textbf{Hardware Constraints:} Training large-scale transformer models on full-length essay data can be computationally expensive. Model size and training configuration may be constrained 
    by available GPU access and memory limitations.
    \item \textbf{Dataset Quality and Scope:} The project relies on publicly available datasets, which may not reflect the full diversity of student writing, prompts, or demographic characteristics. 
    Some datasets may lack metadata needed for fairness analysis.
    \item \textbf{Time Constraints:} The project is constrained by the fixed MSc dissertation timeline. As such, extensions such as feedback generation, multi-modal input (e.g., handwriting), 
    or real-time deployment will only be explored if time permits.
    \item \textbf{Ethical and Legal Restrictions:} Use of student-generated content is limited to datasets with appropriate licenses or ethical clearance. No private or institutionally sensitive data 
    will be used without explicit approval.
    \item \textbf{Model Interpretability Trade-offs:} Highly performant models may be less interpretable. There is a trade-off between achieving high predictive accuracy 
    and ensuring the system remains explainable and educationally useful.
\end{itemize}



\end{document}
